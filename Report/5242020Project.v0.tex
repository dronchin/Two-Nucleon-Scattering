\documentclass[11pt]{article}

\usepackage{bm}
\usepackage{slashed,amsmath,amssymb,fullpage}
\usepackage{xcolor}
\usepackage{color}
\usepackage{graphicx,epsfig}
\usepackage{hyperref}

\newcommand{\calL}{\mathcal{L}}
\newcommand{\calG}{\mathcal{G}}
\newcommand{\Tr}{\text{Tr}}
\newcommand{\be}{\begin{enumerate}}
\newcommand{\ee}{\end{enumerate}}
\newcommand{\plusket}{\vert + \rangle}
\newcommand{\minusket}{\vert - \rangle}
\newcommand{\plusbra}{\langle + \vert}
\newcommand{\minusbra}{\langle - \vert}
\newcommand{\pprime}{p^\prime}
\newcommand{\xprime}{x^\prime}
\newcommand{\aprime}{a^\prime}
\newcommand{\adprime}{a^{\prime\prime}}
\newcommand{\wc}{\omega_c}

\newcommand{\ket}[1]{\vert #1 \rangle}
\newcommand{\bra}[1]{\langle #1 \vert}
\newcommand{\braket}[2]{\langle #1 \vert #2 \rangle}
\newcommand{\id}{\mathbbm{1}}
\newcommand{\half}{\frac{1}{2}}


\begin{document}

\begin{center}
{\large{
{\bf 
Computational Project \\
due Monday May 4th 2020 at 10:00 am
}
}
}
\end{center}


\hrulefill
\medskip

\begin{center}
 {\bf Two-nucleon scattering by a central potential }
\end{center}

Consider two nucleons each of mass $m=938$ MeV/$c^2$
interacting via a central potential given by

$$
V(r) = 
\begin{cases}
              -V_0\, \quad\quad           & \text{for }                r \leq R \,  \\
{\color{white}-}0 \, \quad\quad           & \text{otherwise }\,  \\
\end{cases}
$$

\noindent
where $V_0$ is positive and $R=1.45$ fm. 

\begin{enumerate}
 \item Set $V_0=20$ MeV and determine the $S$, $P$, and $D$-wave phase shifts 
 as function of the energy $E$ in the range of $0-200$ MeV, and plot them. 
 \item Plot the radial wave functions, $u_\ell(r)$, for few representative 
 energies, {\it e.g.}, $E=1$, $10$, $100$, and $200$ MeV.
 \item Evaluate the total cross sections at these energies. 
 \item Plot the total and partial cross sections as function of the energy. 
  \item The $S$-wave phase shift can be calculated analytically. Use the analytical 
 solution of the phase shift and the cross section in $S$-wave to validate your code.
 \item Repeat the steps above with $V_0=60$ MeV. Note that there is an ambiguity
 in the definition of the phase shifts. This will make your phase shift in $S$-wave
 discontinuous when approaching $\pi/2$. Make your phase shift continuous by adding
 the appropriate phase. 
 \item Write a report describing your code and its validation against the analytical
 solutions, and your findings. If you have time, you may take this opportunity to 
 practicing with Latex. If you are working in pairs and writing the report together 
 you may find \href{https://www.overleaf.com/}{overleaf} useful. 
\end{enumerate}

If you are working with a colleague please try to each contribute to the final product. 
If this is the first time you are coding, take this opportunity to get exposed 
to and play with coding, and contribute to the project by, {\it e.g.}, working out the
analytical solutions, writing the project report, working on the figures 
and so on. In practice, work {\it together} (each at their respective homes). 

\begin{center}
{\bf 
Implementation
}
\end{center}

We seek the solution, $u(r)=r\,R(r)$, of the radial equation
\begin{equation}
 -\frac{\hbar^2}{2\,\mu}u^{\prime\prime}(r) +V(r)\,u(r) +\frac{\ell(\ell+1)}{r^2}=E\,u(r)\, ,
\end{equation}
where 
\begin{eqnarray}
 E&=&\frac{\hbar^2\,k^2}{2\,\mu}\, , \quad {\rm and}\quad \quad  \mu=\frac{m}{2} \, .
\end{eqnarray}
We define
\begin{equation}
 v(r)=\frac{2\,\mu}{\hbar^2}\,V(r) \, ,
\end{equation}
and rewrite the equation above  as
\begin{eqnarray}
&& u^{\prime\prime}(r) +\left[k^2 - v(r)\,u(r) -\frac{\ell(\ell+1)}{r^2}\right]\,u(r)=0\, ,\nonumber 
\end{eqnarray}
or
\begin{eqnarray}
&& u^{\prime\prime}(r) + K(r) u(r)=0 \, ,
\end{eqnarray}
with 
\begin{equation}
 K(r)=\frac{2\,\mu}{\hbar^2}\left[E-V(r)\right] - \frac{\ell(\ell+1)}{r^2} \, .
\end{equation}

As $r\rightarrow 0$, $u(r)\rightarrow r^{(\ell+1)}$. We use two grid points 
close to zero ({\it e.g.}, $r_1=h$ and $r_2=2\times h$, where $h$ is the 
step in $r$), and the solution calculated in these points, that is 
$u_1=u(r_1)=h^{\ell+1}$ and $u_2=u(r_2)=(2\times h)^{\ell+1}$, to start building the
solution outwards using the Numerov's method. 

The subsequent $u_i$'s calculated at $r_i$'s are obtained using the following
algorithm
\begin{equation}
u_{i+1}\left(1+\frac{h^2}{12}K_{i+1} \right) -u_i\,\left(2-\frac{5\,h^2}{6}K_i\right)+
u_{i-1}\left(1+\frac{h^2}{12}K_{i-1} \right) +O(h^6)=0 \nonumber \, . 
\end{equation}

TO BE CONTINUED...
\begin{figure}
% \includegraphics[width=3.3in]{yourfigure.pdf}
 \caption{Your figure here.}
\label{fig:yourfigure}
\end{figure}
% 
 
You will need the physical constant $\hbar\,c = 197.326980$ MeV fm.
\end{document}


